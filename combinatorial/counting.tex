%%%
%%% Combinatorics
%%%

\problemtitle{Counting}

\begin{algorithm}{Binomial $\binom{n}{k}$}
\reflisting{choose}
\complexity{\min\{k, n-k\}}
\end{algorithm}

\begin{algorithm}{Multinomial $\binom{\Sigma k_i}{k_1\;k_2\;\ldots\;k_n}$}
\reflisting{multinomial}
\complexity{(\Sigma k_i)-k_1}
\end{algorithm}

\begin{algorithm}{String permutations (multinomial)}
\reflisting{nperms}
\usage{ \sourceline{ string s; int n = n\_perms( s ); } }
\desc
Algorithm for calculating the number of permutations of a string
(multinomial numbers).
\end{algorithm}

\begin{algorithm}{Stirling numbers of the first kind}
\reflisting{stirling1}
\usage{ \sourceline{s = stirling1(n,k);} }
\desc
The Stirling numbers of the first kind $s(n,k)$ is defined as
$(-1)^{n-k}c(n,k)$, where $c(n,k)$ is the number of permutations on
$n$ items with $k$ cycles.
\end{algorithm}

\begin{algorithm}{Stirling numbers of the second kind}
\reflisting{stirling}
\usage{ \sourceline{s = stirling(n,k);} }
\desc
Calculates the stirling number $S(n,k)$, i.e. in how many ways can $n$
different items be put in $k$ boxes with at least one item in every
box, or mathematically speaking -- the number of partitions of $n$
elements into $k$ partitions.
\end{algorithm}

\begin{algorithm}{Stirling numbers of the second kind modulo 2}
\reflisting{stirling mod 2}
\usage{ \sourceline{s = stirling\_mod\_2(n, k);} }
\complexity{ \log{k} }
\end{algorithm}


\begin{algorithm}{Bell numbers}
\desc
$B(n) = \sum_{k=1}^n \binom{n-1}{k-1} B(n-k) = \sum_{k=1}^n S(n,k)$,
where S(n, k) are the Stirling numbers of the second kind.

The Bell numbers count the ways $n$ elements can be partitioned.
\end{algorithm}

\begin{algorithm}{Eulerian numbers}
\reflisting{euler}
\usage{ \sourceline{s = euler(n,k);} }
\desc
The Eulerian number $e_{n,k}$ counts the number of permutations $\pi$
of $[n]$ with
\begin{itemize}
\item $k$ descents/ascents (a descent is a $j$ s.t. $\pi(j) > \pi(j+1)$)
\item $k+1$ weak excedances (a weak excedance is a $j$ s.t. $\pi(j) \ge j$)
\item $k$ excedances (an excedance is a $j$ s.t. $\pi(j) > j$)
\end{itemize}
$$e_{n,k} = (n-k)e_{n-1,k-1} + (k+1) e_{n-1, k} = \sum_{j=0}^{k+1}
(-1)^j \binom{n+1}{j} (k - j + 1)^n$$

{\bf NB!!} It is common to define the Eulerian number as $A(n,k) =
e_{n,k-1}$ (this stems from the fact (or does it?) that the $n$:th
Eulerian polynomial is $A_n(x) = \sum_{k=1}^{n} A(n,k)x^k$).
\end{algorithm}

\begin{algorithm}{Second-order Eulerian numbers}
\reflisting{euler2}
\usage{ \sourceline{s = euler2(n,k);} }
\desc
The second-order Eulerian number $e_{nk}$ is the number of
permutations $\pi_1 \pi_2 \cdots \pi_{2n}$ of the multiset
$\{1,1,2,2,\cdots,n,n\}$ with the property that all numbers between
the two occurences of $m$ are greater than $m$ that have $k$ places
where $\pi_j < \pi_{j+1}$.
\end{algorithm}

\begin{algorithm}{Catalan numbers}
\keyword{}
\desc
Among other things, the Catalan numbers describe the number of ways a polygon
with n+2 sides can be cut into n triangles, the number of ways in which
parentheses can be placed in a sequence of numbers to be multiplied, two at
a time; the number of rooted, trivalent trees with n+1 nodes; and the number
of paths of length 2n through an n-by-n grid that do not rise above the
main diagonal.
$$C_n = \frac{2(2n-1)C_{n-1}}{n+1} = \frac{\binom{2n}{n}}{\scriptstyle n+1}$$
\end{algorithm}

\begin{algorithm}{Derangements}
\keyword{}
\desc
A permutation that leaves no element in its original position.  The
number of such permutations on $[n]$ is given by
$$D_n = (n-1)(D_{n-1}+D_{n-2}) = nD_{n-1} + (-1)^n = n!\left(\frac
1{2!}-\frac 1{3!}+\ldots+(-1)^n\frac 1{n!}\right) =
\left[\frac{n!}{e}\right]$$
This number is sometimes called the subfactorial, denoted $!n$.
\end{algorithm}

\begin{algorithm}{Involutions}
\keyword{}
\desc
An involution is a permutation with maximum cycle length 2, or
equivalently, a permutation which is its own inverse.  The number of
involutions on $[n]$ is given by
$$s(n) = s(n-1) + (n-1)s(n-2) \qquad s(0) = s(1) = 1$$
\end{algorithm}

\begin{algorithm}{Permutation statistics}
\keyword{}
\desc
Let $f: S_n \rightarrow \mathbb{Z}_{\ge}$ be a function describing
some property of a permutation (e.g. the number of inversions), and
$F(n,k) = \#\{\pi | f(\pi) = k\}$.  Then the polynomial
$$\sum_{\pi \in S_n} x^{f(\pi)} = \sum_{k} F(n,k) x^k$$ 
is called the permutation statistic (c.f. the Eulerian polynomials)
for the property. We have
\begin{eqnarray*}
\sum_{\pi \in S_n} x^{c(\pi)} & = & \sum_{k} c(n,k) x^k = x (x+1) \cdots (x+n-1)\\
\sum_{\pi \in S_n} q^{i(\pi)} & = & (1 + q)\cdot(1+q+q^2) \cdots (1 + q + \cdots + q^{n-1})
\end{eqnarray*}
where $c(\pi)$ is the number of cycles of $\pi$, and $i(\pi)$ is the
number of inversions of $\pi$.
\end{algorithm}
