\problemtitle{Number Theory}

\begin{desc}
Miscellaneous algorithms in number theory.
\end{desc}

\begin{algorithm}{GCD}
\usage{ \sourceline{d = gcd( a, b );} \quad\quad $a>b$ }
\complexity{ \log(b) }
\characteristics{}
\begin{example}
	gcd( 10, 6 ) == 2
\end{example}
\valladolid{202}
\end{algorithm}


\begin{algorithm}{Euclid}
\usage{ \sourceline{d = euclid( a, b, x, y );} \quad $a>b$, %
	$x$ and $y$ are return values that satisfy $ax+by=d$. }
sadsad

asdsadasasdsadas asdsadasasdsadas asdsadas asdsadas asdsadas asdsadas asdsadas
dfdfasdsadasasdsadas asdsadasasdsadas asdsadas

\complexity{ \log(b) }
\characteristics{$x$ and $y$ have (hopefully, probably, don't know...) the smallest absolute value}
\begin{example}
	euclid( 10, 6, x, y ) == 2, (x,y)==(-1,2)
\end{example}
\valladolid{202}
\end{algorithm}

\begin{sources}
\code{gcd}{gcd}
\code{euclid}{euclid}
\end{sources}