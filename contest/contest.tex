\categorytitle{Contest}
\categorycontents{}

\problemtitle{Mandatory}

\begin{algorithm}{Template}
\reflisting{Template}
\desc
Standard problem template. Problems are classified as either of:
  \begin{description}
  \item{simple-solve} -- Number of test cases is 1.
  \item{for-solve} -- Number of test cases is given in the input.
  \item{while-solve} -- End of test cases is indicated by special values
    or end of input. Stop when {\tt solve} returns false.
  \end{description}
\end{algorithm}

\begin{algorithm}{Script}
  \reflisting{script}
  \usage{\sourceline{ sh script }}
  \begin{description}
  \item{c} -- Compile
  \item{i} -- Enter program input
  \item{o} -- Enter correct program answer
  \item{t} -- Test using entered program input
  \item{td} -- Test with direct typed input
  \item{d} -- Diff the program output with the entered correct answer
  \item{p} -- Print source code
  \item{submit} -- Submit a solution!!
  \item{n} -- New problem, copy Template
  \item{f} -- Finished problem, move to done
  \end{description}
\end{algorithm}

\begin{algorithm}{Emacs Key Bindings}
\reflisting{contest-keys.el}
\reflisting{contest-extras.el}
\usage{\sourceline{ M-x load-file RET contest-keys.el }}
\usage{\sourceline{ M-x load-file RET contest-extras.el }}
\desc
contest-keys.el provides useful key bindings for Emacs.
\begin{description}
\item{C-x C-f} -- New file (overloads the usual find-file, uses Template.cc)
\item{C-c c} -- Compile (uses compile command as specified in c-lite.el)
\item{C-c t} -- Test solution (using ``\texttt{FILE < FILE.in}''. The
commented line is for testing using ``\texttt{FILE}'', i.e. when files
are used instead of stdio)
\item{C-c s} -- Submit solution (using ``\texttt{submit FILE.cc}'')
\end{description}
contest-extras.el contains some additional key-bindings
\begin{description}
\item{C-c p} -- Print current file with line numbers.
\item{C-c d} -- Run program on input and diff with output.
\item{C-c g} -- Goto line (useful for fixing compilation errors).
\end{description}
\end{algorithm}


\begin{sourceslandscape}

\sourcesection{Mandatory Contest Material}
\code{Template}{Template}
\code{script}{script}
\code{contest-keys.el}{contest-keys.el}
\code{contest-extras.el}{contest-extras.el}

\end{sourceslandscape}
